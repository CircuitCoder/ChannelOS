\documentclass[UTF-8]{ctexbeamer}
\usetheme{Copenhagen}
\usecolortheme{seahorse}

\usepackage{multimedia}
\usepackage{listings}
\usepackage{minted}
\usepackage{tikz}
\usepackage{textcomp}
\usepackage[normalem]{ulem}

\title{动态内核模块放置 - Midterm}

\author{刘晓义}
\date{2022.10.13}

\begin{document}

\part{Intro}

\begin{frame}
  \titlepage
  \begin{center}
    \includegraphics[width=.1\textwidth]{assets/float.png}
  \end{center}
\end{frame}

\begin{frame}[fragile]
  \frametitle{Motivation}

  微内核是好的,但是延迟大。

  \pause

  宏内核也是好的,但是 Syscall 慢。

  \pause

  \vspace*{1em}

  Linux 内核模块可以选择动态加载和编译进内核
  
  \pause
  
  能否添加一种将模块可选放置在用户态的办法?

  \begin{minted}{bash}
modprobe some_mod
modprobe --user-process another_mod
modprobe --sync-caller and_another_mod
  \end{minted}
\end{frame}

\begin{frame}
  \frametitle{Approach}

  \begin{itemize}
    \item 需要一个办法能够在用户态进程之间传递消息
    \item 需要一个对用户代码透明的办法,在不同消息传递方法之间切换
  \end{itemize}

  \pause
  \vspace*{1em}

  \begin{itemize}
    \item 建立一个类似 C/S 结构的服务模型
    \begin{itemize}
      \item A fancy way to say io\_uring
    \end{itemize}
    \item 所有模块和用户态进程都声明自己的提供的\textbf{服务}和需要的\textbf{依赖}
    \item 内核在加载代码段的时候动态决定调用的具体方式: similar to vDSO
  \end{itemize}
\end{frame}

\begin{frame}[fragile]
  \frametitle{Approach, Cont.}

  \small
  \begin{minted}{c}
/**
 * Service handler,
 * - May be called by event queue.
 * - May be called directly if applicable.
 */
uint64_t meow_service_handler(message *msg);

REGISTER_SERVICE(
  0x8080808080808080ull, // Service Identifier (syscall number)
  meow_service_handler,
);

void init() {
  // Setup process states...

  // Event queue implementation provided by kernel
  INIT_SERVICES();
}
  \end{minted}
\end{frame}

\begin{frame}[fragile]
  \frametitle{Approach, Cont.}
  \begin{minted}{rust}
    let meow_service(args).await!;
  \end{minted}

  \pause
  \vspace*{1em}

  也许会被实现成:
  \begin{itemize}
    \item 直接调用(对应模块和本模块在同一地址空间,例如同时在内核中,并且要求低延迟)
    \item 同步 syscall(对应模块在内核中,本模块在用户态中,并且要求低延迟)
    \item Ring(对应模块以进程形式执行,无论是在内核还是在用户态)
  \end{itemize}

  \pause
  \vspace*{1em}

  实现使用 vDSO 注入
\end{frame}

\begin{frame}
  \frametitle{Challenges}

  如何实现一个高效的 Ring,尽量减少内核参与?

  \pause

  \begin{itemize}
    \item 用户态中断
    \item 共享内存 + 一些额外唤醒机制
  \end{itemize}
\end{frame}

\begin{frame}[fragile]
  \frametitle{Trap fast path}

  传统 Signal :保存现场,切换到内核栈,然后进入内核代码处理 Syscall。完成后恢复现场,切换回用户栈。

  我们希望唤醒操作尽量快

  \pause
  \vspace*{1em}

  保留一个页存储所有被唤醒的目标进程,当下一次 Scheduler 被触发的时候执行(irq\_work?):

  \begin{minted}{c}
  #define WORKQUEUE_CAP ((4096 / 8) - 1)
  struct WakeupQueue {
    uint64_t cur = 0;
    uint64_t slots[WORKQUEUE_CAP];
  };
  \end{minted}

  sscratch 保存一个 Struct 的指针:

  \begin{minted}{c}
  struct BackgroundContext {
    void *kstack_top;
    WorkQueue *wake_queue;
    // Other important stuff
  }
    
  \end{minted}

\end{frame}

\begin{frame}[fragile]
  \frametitle{Apprach, Cont.}

  \small

  \begin{columns}
    \begin{column}{0.5\textwidth}
      \begin{minted}{asm}
trap_enter:
  beq a0, x0, wakeup_fast_path
  // ...
wakeup_fast_path:
  csrrw t0, sscratch, t0 
  beq t0, x0, trap_cont

  STORE(t1, 0)
  STORE(t2, 1)
  csrr t1, scause
  bneq t1, U_CALL, trap_cont

  ld t1, WAKEUP_PAGE_BASE(t0)
  li t2, 1
  amoadd.d t2, t2, (t1)
  subi t2, t2, WORKQUEUE_CAP
  blt t2, x0, wakeup_full
      \end{minted}
    \end{column}
    \begin{column}{0.5\textwidth}
      \begin{minted}[]{asm}
  addi t2, t2, WORKQUEUE_CAP
  slli t2, t2, 3

  addi t2, t1, t2
  sd a1, (t2)

  LOAD(t1, 0)
  LOAD(t2, 1)
  csrrw t0, sscratch, t0
  sret
      \end{minted}
    \end{column}
  \end{columns}
\end{frame}

\begin{frame}
  \frametitle{Progress}
  \begin{itemize}
    \item 实现了上述 Ring 和 sched 配套的唤醒机制
    \item 定义了 Syscall 接口
    \begin{itemize}
      \item 0: 唤醒
      \item 1: WFI
      \item 2: 同步调用其他服务
    \end{itemize}
    \item 定义了一系列内核提供的服务
    \item 现在用户态进程可以\textbf{显式}异步调用服务:
    \begin{itemize}
      \item 串口输出,位于用户态
      \item brk 内存分配,由内核提供
    \end{itemize}
  \end{itemize}
\end{frame}

\begin{frame}
  \frametitle{Plan}
  
  为了保证对用户代码透明,所有服务调用都返回 Future(如果被配置成同步调用,会返回一个 Resolved Future)

  内核需要根据服务调用者和被调用者之间的相对位置关系,和用户配置,决定链接什么调用代码。

  \begin{itemize}
    \item 直接函数调用方法类似动态链接(PLT indirect call)
    \item 异步调用在 PLT 基础上需要链接 Ring 相关的管理逻辑
  \end{itemize}

  \pause
  \vspace*{1em}

  注意到,同一个模块的所有下游使用者可以共享相同的 Ring 管理逻辑

  \begin{itemize}
    \item 在编译模块的时候额外添加一个段放置这些代码(.remote.text),是由 \texttt{REGISTER\_SERVICE} 生成的。
    \item 加载模块的时候在内核中注册
    \item 服务的使用者在加载时将上述代码链接进去,并且直接把 PLT 指向那里
  \end{itemize}
\end{frame}

\begin{frame}
  \frametitle{Plan, Cont.}
  接下来需要干的事情:

  \begin{itemize}
    \item 动态链接基础设施
    \item vDSO 实现
    \item \texttt{.remote.text} 代码生成
  \end{itemize}
\end{frame}

\begin{frame}
  \frametitle{That's All!}
  \begin{center}
    \includegraphics[width=.5\textwidth]{assets/look.png}

    Question time!
  \end{center}
  \url{https://github.com/CircuitCoder/ChannelOS}
\end{frame}

\end{document}