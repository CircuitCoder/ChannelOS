\documentclass[UTF-8]{ctexbeamer}
\usetheme{Copenhagen}
\usecolortheme{seahorse}

\usepackage{multimedia}
\usepackage{listings}
\usepackage{minted}
\usepackage{tikz}
\usepackage{textcomp}
\usepackage[normalem]{ulem}
\title{动态内核模块放置}

\author{刘晓义}
\date{2022.11.10}

\begin{document}

\part{Intro}

\begin{frame}
  \titlepage
  \begin{center}
    \includegraphics[width=.1\textwidth]{assets/float.png}
  \end{center}
\end{frame}

\begin{frame}[fragile]
  \frametitle{回顾}

  能不能在运行时决定内核提供的一个服务运行在哪里,并且

  \begin{itemize}
    \item 有足够好的性能
    \item 无需更改用户程序
  \end{itemize}

  \vspace*{3em}
  \url{https://github.com/CircuitCoder/ChannelOS}
\end{frame}

\begin{frame}
  \frametitle{做出来的饼}

  \begin{itemize}
    \item 异步通讯协议
    \item ABI / API
    \item Dynamic linker / vDSO 框架
  \end{itemize}
\end{frame}

\begin{frame}
  \frametitle{没做出来的饼}

  \begin{itemize}
    \item SMP
    \item Codegen
  \end{itemize}
\end{frame}

\begin{frame}
  \frametitle{计划外的事情}

  vDSO 框架抽取成单独的模块

  \url{https://github.com/CircuitCoder/kernel_prelink}
\end{frame}

\begin{frame}
  \frametitle{如果后续继续做的话}

  \begin{itemize}
    \item 把 SMP 支持调通
    \item 实现一个网络栈
    \item 硬件上性能测试
  \end{itemize}
\end{frame}

\end{document}